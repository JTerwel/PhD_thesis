%!TEX root = ../main.tex
\documentclass[a4paper,oneside,12pt, class=Latex/Classes/PhDthesisPSnPDF, crop=false]{standalone}
\usepackage{setspace}
\begin{document}
\doublespacing
\chapter{Introduction}
\label{chap:intro}

\section{The final stages of stars}

\begin{figure}
    \centering
    \includegraphics[width=0.49\textwidth]{../Images/chapter_1/SN2024gy_pre-SN.png}
    \includegraphics[width=0.49\textwidth]{../Images/chapter_1/SN2024gy_active.png}
    \caption{\ztfg\ztfr\ztfi\ composite image of NGC4216 using observations taken by the Zwicky Transient Facility. \textbf{Left:} composite image of approximately \color{red}XX \color{black} observations in each filter, taken before 1 January 2024. \textbf{Right:} composite image of approximately \color{red}YY \color{black} observations in each filter, taken between 5 and 19 January 2024, the first two weeks after the first detection of the Type Ia SN~2024gy. (Credit: Benjamin Nobre Hauptmann) \color{red}Use this as an example when introducing transients \color{black}} %30, 29, 14 (gri sn images) & 35, 31, 30 (gri pre-SN images) given to Benjamin, all used or a subset?
    \label{2024gy_ZTF}
\end{figure}

\section{Type Ia SNe}


\end{document}