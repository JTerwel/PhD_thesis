%!TEX root = ../main.tex
\documentclass[a4paper,oneside,12pt, class=Latex/Classes/PhDthesisPSnPDF, crop=false]{standalone}
\usepackage{setspace}
\begin{document}
\doublespacing
\chapter{Conclusions and future work}
\label{chap:Conc_and_fut}

In this thesis I have presented my work on the search for signs of interaction between the ejecta of SNe Ia and distant CSM. Due to the distance of the proposed CSM the interaction would start at late times, which I defined to be $\geq100$ days after the observed SN peak in the observer frame, and cause an extreme flattening or rebrightening in the SN light curve. I searched through three different samples of objects, these being the ZTF SN Ia DR2 (section \ref{chap:DR2_search}), transients that were first detected between 2008 and 2018, including objects that were not SNe Ia (section \ref{chap:pre-ZTF_search}), and a real-time search using all SNe Ia discovered by ZTF up to July 2023 in order to follow candidates up using the NOT and GTC (section \ref{chap:Real-time}).

To conduct this search in a systematic manner and to be able to recover the faintest signals hidden within the data, I built a custom analysis pipeline, which is presented in section \ref{pipeline} and tested through a simulated observing campaign in section \ref{simulation}. After an initial cleaning and calibration of the data, the late-time observations were binned together in bins with a size between 25 and 100 days. By doing so, I was able to push the detection limit beyond that of the individual observations, near the noise limit of the reference images used by ZTF in their difference imaging pipeline. This was done separately for the different ZTF bands \ztfg\ztfr\ztfi.


\section{Conclusions}
Searching through these three samples made one thing abundantly clear: late-time CSM interaction in SNe Ia is very rare, and the only way to systematically find them is by searching through large data sets, preferably catching them while they are active so they can be followed up on. By doing so, it will be possible to study properties of the CSM, such as its size and composition, gain new insights in SN Ia explosions and the physics that governs them, and place constraints on their progenitor systems. Based on the work I presented, my main conclusions are:

\begin{enumerate}
	\item Across the three samples I analysed, the late-time data consisted of 12\,601 unique spectroscopically confirmed Type Ia SNe, and an additional 3\,089 transients in the pre-ZTF sample that were not classified as SN Ia or one of its subclasses. For some of these transients the data extended up to 15 years after they were first discovered. Out of these, I identified seven candidate objects showing signs of possible late-time CSM interaction, with the first detection of their late-time signal being between 250 and 2\,200 days after discovery in the SN rest frame. Additionally, SN 2020qxz was spectroscopically confirmed to have a period of rebrightening around 1\,150 days after peak light in the SN rest frame.
	\item The largest fraction of objects that are flagged by the pipeline are false positives that show spurious detections in their binned light curves. This is a result of the wide net it casts in an attempt to minimize the chance of missing a faint signal. The visual inspection of these light curves, as well as inspecting the difference images using \textsc{snap}, showed that in most cases the spurious detections are either due to image issues (e.g., in a difficult environment) or due to software issues (e.g., imperfect image subtractions).
	\item Another large group of objects that were flagged by the pipeline are bright nearby SNe, whose declining tail could still be recovered in the bins. Despite attempting to filter for these cases, some slipped through due to gaps in the observations or a tail that did not follow the simple decline model used in the test. These included most known SNe Ia-CSM in the sample and several very nearby objects whose decline displayed a kink around a year after the peak due to a change in their opacity.
	\item I also recovered over a dozen pairs of sibling transients, two transients at (nearly) the exact same location. In one case, this is a line-of-sight effect, with AT 2018iml likely being a CV in the foreground of SN 2009hz. The other cases were sibling SNe occurring in the same galaxy. Some of these were already known, with both siblings already having been classified. In other cases, a possible but unconfirmed (extinct) sibling transient fits the late-time signal. While such sibling transients are not a very common occurrence, modern surveys discover enough new transients that these rare events can be found regularly.
	\item Most objects with a late-time signal that was not easily recognized as a false positive, SN tail, or sibling transient, were close to or on top of the nucleus of their host galaxy. These are difficult and busy environments, and even if the nucleus is not active, there is a lot of room to interpret a late-time signal as something related to the host nucleus instead. One class of possible contaminants I considered are ANTs, ambiguous nuclear transients that do not really fit in the AGN or TDE classes. The objects I recovered are broadly consistent but generally fainter than most known ANTs, suggesting the existence of a previously unknown population of faint ANTs.
	\item Transients that were first detected before the start of ZTF but still visible after the survey started were never properly detected by ZTF, as they are in the reference images that were made shortly before the survey started in March 2018. As a result of this, difference imaging will leave a ghost, a negative imprint that is easily identified using \textsc{snap} (see section \ref{snap}). Forced photometry still works, but the measured flux will be too small or even negative. Through a proper baseline correction, as is done at the start of my pipeline, these objects can still be recovered. AGN and variable stars can be recovered in the same way, though a proper baseline correction is more difficult.
	\item Between the three samples, I recovered seven candidate objects, with their late-time signals starting between 0.7 and 5.9 years after the SN, lasting between $\sim100$ days to over a year, and having an absolute magnitude of M~$\sim-16$~mag or brighter. In section \ref{CSM_calc}, I showed a first-order estimate for CSM mass based on broadband magnitude, assuming that the main emission mechanism is bremsstrahlung. Through this, I showed that it is possible to have interaction years after the SN that is this bright, but it requires a large CSM mass in a thin shell. Due to the size of the shell and the finite speed of light, even a short interaction will be smeared out over a long period of time.
	\item Through a quick detection and fast follow-up of SN 2020qxz, I was able to spectroscopically confirm the late-time signal through four emission lines that had disappeared a month later. This object stands out from the candidates for several reasons, as it is a known Ia-CSM that showed signs of interaction near its peak, had a very short period of late-time interaction that was detectable with ZTF, and was visually offset from its host galaxy (making spectroscopic follow-up much easier). The emission lines appear very blueshifted ($5\,150 - 5\,920$~km~s$^{-1}$), and include \Hbeta\ emission. This could suggest that the second episode of CSM interaction is between accelerated material from the first episode of CSM interaction and a cloud of CSM on our side of the SN.
	\item With the method that I used in this thesis to find faint late-time signals, even a real-time search is not completely in real time, but lags a few days to weeks behind. This is because several observations are needed before a newly appearing late-time signal can cause its bin to rise significantly above the noise. The need for several new observations can, however, be used to limit the amount of processing that has to be done on a daily basis. There will always be a trade-off between catching things quickly and catching them while they're faint, which can result in situations like that of SN 2020qxz, where follow-up was only possible for the fading part of the signal.
	\item In section \ref{rates_csm} I estimated that the rate of strong late-time interaction in SNe Ia is $<0.5$\% of SNe Ia, which translates to an absolute rate between $8_{-4}^{+20}$ and $54_{-26}^{+91}$ Gpc$^{-3}$ yr$^{-1}$ when assuming a constant SN Ia rate of $2.4\times10^{-5}$ Mpc$^{-3}$ yr$^{-1}$ for $z \leq 0.1$. While the pre-ZTF and real-time samples did not allow me to make separate robust estimates of this rate, the low number of recovered (candidate) objects confirms the rarity of late-time CSM interactions in SNe Ia.
\end{enumerate}


\section{Future work}
\subsection{Improvements to the pipeline}
One thing that has become very clear, and is commented on in the conclusions of chapters \ref{chap:DR2_search}, \ref{chap:pre-ZTF_search}, and \ref{chap:Real-time} is the limitations that the depth of ZTF places on my ability to recover such faint signals, even when using techniques like light curve binning. The upcoming Vera C.~Rubin Observatory's Legacy Survey of Space and Time \cite[LSST;][]{LSST} will be able to detect objects several magnitudes deeper than ZTF, allowing the detection of fainter objects in a larger volume. With this in mind, there are several improvements that can be made to the pipeline in preparation for a similar search using LSST. This has the advantage that if this is done before the survey starts, everything can be done in real time with possible follow-up opportunities. Possible samples to monitor from the start of LSST include the ZTF SN Ia DR2 or a similar well-understood sample from ZTF that includes other types of transients and transients discovered after the start of 2021. Possible improvements to the pipeline include:

\begin{itemize}
	\item Integrating the AGN check based on WISE data with the binning program, automatically generating \textsc{snap} movies, and other adaptations to streamline and automate the analysis process, making it easier for a real-time search.
	\item Lessen the amount of false positives flagged by the pipeline, and improve the methods for detecting SN peaks and removing bright SN tails.
	\item Generalize the procedure to work for other types of transients. One of the reasons why only the pre-ZTF sample included non-SN Ia transients was that the tail removal procedure was not built to handle those objects and could not be used.
	\item Machine learning. Different classes of objects recovered with the pipeline have different kinds of binned signals. Machine learning can streamline, automate, and speed up the classification of signal types and possibly identify new classes.
\end{itemize}


\subsection{Alternative uses for the pipeline}
As has been shown repeatedly throughout this thesis, the binning procedure is a catch-all method that is able to find any kind of faint signal. This is one of the reasons why signals coming from, e.g., declining SN tails, siblings, and nuclear transients are recovered even though the pipeline is not optimized for them. This means that the method and pipeline can easily be adapted for other types of transients or research goals, such as:

\begin{itemize}
	\item Declining SNe to follow their evolution out to later epochs.
	\item Sibling transients to learn more about the environmental effects on SNe. The current pipeline only finds sibling transients when the first one is the brightest. To find sibling transients that are spatially far enough apart not to have one leave an imprint in the forced photometry light curve of the other transient, a different pipeline will be needed (e.g., SN~2024nqr and SN~2024pgd, see Fig.~\ref{phot_spec_example}).
	\item (faint) ANTs and other nuclear transients. This could use the pipeline version that was used for the pre-ZTF sample in chapter \ref{chap:pre-ZTF_search}, utilizing the choice of different baseline regions. All that is needed is to define a new sample of positions to generate forced photometry light curves at.
	\item Precursor events. Everything that I have done and everything that can be done to look at what happens after the SN can be reversed to check what happened before the SN. This can, for instance, be done for LSST transients to see if ZTF observed any signs of precursor activity.
	\item Use in other surveys. There are many other surveys, both active and complete, whose data could be processed with my pipeline, as long as the data structure is transformed to what the code expects. ASASSN, ATLAS, GOTO \citep[Gravitational-wave Optical Transient Observer,][]{GOTO_prototype, GOTO}, and (i)PTF are just a few examples. This could be especially powerful if data from multiple surveys is used together to create long and/or well-observed light curves.
\end{itemize}


\subsection{Other future work}
Besides future work that involves using an adapted version of the binning procedure or pipeline, there are more questions to be answered based on the results I presented in this thesis. Some examples of these are:

\begin{itemize}
	\item A proper CSM mass estimate using broadband observations. While the best estimates will rely on the emission lines that can be spectroscopically confirmed, if it is possible to get a decent CSM mass estimate given a certain assumption for the composition and state of the CSM, this should be useful for getting a first estimate or in cases where no spectroscopic follow-up was possible. The attempt I made in section \ref{CSM_calc} is a good step, but it can be improved upon.
	\item In section \ref{2020qzx_discussion}, I propose a scenario that could explain the large velocity offsets that I found in the late-time transient emission lines of SN 2020qxz. Whether or not such a scenario is actually viable, and how common the right conditions would be to generate such offset emission lines, will require detailed modelling and simulations.
\end{itemize}

Modern large-scale surveys that observe the night sky and monitor its changes on a regular basis generate a wealth of data that can be mined for years. By combining observations that have been taken over timespans of years, it is possible to follow the evolution of transients for much longer timescales than they are usually studied. Binning observations together is a powerful way to detect the faintest signals and can be used to find new and rare events that may or may not be connected to brighter types of events that are already known. Through this, it will be possible to find new clues to help us understand some of the most energetic events in the universe and the physical processes that govern them.

\end{document}