%!TEX root = ../main.tex
%Adding the above line, with the name of your base .tex file (in this case "main.tex") will allow you to compile the whole thesis even when working inside one of the chapter tex files


\newgeometry{left=2.5cm, right=2.5cm, top=2.0cm, bottom=2.0cm, footskip=1cm}
\begin{abstracts} 
The nature of the progenitor systems and explosion mechanisms that give rise to Type Ia supernovae (SNe Ia) are still debated. In rare cases the ejecta interact with circumstellar material (CSM) ejected from the progenitor system before the explosion. As it is being swept up by the expanding ejecta the CSM starts to glow and becomes an additional light source that can be bright enough to drastically alter the SN evolution. By studying the interaction signature the properties and composition of the CSM can be revealed, which can constrain the type of progenitor system it was ejected from. However, most previous studies have focused on finding CSM ejected shortly before the SN Ia explosion, which still resides close to the explosion site resulting in short delay times until the interaction starts.

In this thesis I search data from the Zwicky Transient Facility (ZTF) for signs of late-time interaction between SN Ia ejecta and distant CSM, where interaction starts $>100$ days after the explosion. I use SN 2015cp, where interaction was discovered 664 days after peak brightness, as a prototype event and develop a pipeline to search for late-time rebrightening events in ZTF forced photometry light curves. By binning the late-time light curve data, I push the detection limit as deep as possible and recover faint signals below the detection limit of individual data points.

I search through three samples of transients: 1) The ZTF second data release (ZTF SN Ia DR2), which containts all spectroscopically classified SNe Ia discovered by ZTF from March 2018 to October 2020. 2) The pre-ZTF sample, which consists of all transients discovered between 2008 and 2018. 3) All SNe Ia discovered by ZTF before 8 July 2023, whose late-time evolution I monitor from 13 November 2023 to 9 July 2024 to find active late-time rebrightening events and follow these up with secondary telescopes. In total, I analyze 12\,601 unique spectroscopically confirmed Type Ia SNe and 3\,089 other transients. Out of these I identify seven SNe Ia that show signs of possible late-time CSM interaction starting between 250 and 2\,200 days after the explosion and lasting between 100 and 500 days. I also discovered a short ($\sim50$~d) signal in SN 2020qxz, which I spectroscopically confirmed to be a period of rebrigthening around 1\,150 days after peak light in the SN rest frame. I also recover several other types of objects, including sibling transients, declining tails of bright SNe, and a group of (faint) nuclear transients. This shows my pipeline's ability to different types of faint signals that are only recoverable through binning.

Using the well-defined nature of the ZTF SN Ia DR2 and simulations of the survey I estimate that $>0.5$ per cent of normal SNe Ia display late-time ($> 100$~d post peak) strong \Halpha-dominated CSM interaction. This is equivalent to an absolute rate of $8^{+20}_{-4}$ to $54^{+91}_{-26}$ Gpc$^{-3}$ yr$^{-1}$, assuming a constant SN Ia rate of $2.4 \times 10^{-5}$ Mpc$^{-3}$ yr$^{-1}$ for $z \leq 0.1$. Weaker interaction signatures of \Halpha\ emission, more similar to the strength seen in SN 2015cp, could be more common but are difficult to constrain with the depth of ZTF. By using a simple CSM model with brehmsstrahlung as the main emission mechanism during the interaction, I show that interaction with a thin, distant shell of CSM containing $\lesssim5$ M$_\odot$ of material can produce a signal that is detectable with ZTF, even if the interaction starts up to six years after the SN.

Most of the identified candidates reside close to the nuclei of their host galaxies, which suggests that environment or specific progenitor characteristics play a role in the production of the potential late-time CSM signatures in these SNe Ia. The exception to this is SN 2020qxz, a SN Ia with an earlier period of CSM interaction as well as a short late-time signal. Through spectroscopic followup I confirm four transient emission lines which I associate to \Hbeta, \CaII, \NI, and \KI, and are blueshifted by $5\,150 - 5\,920$ km~s$^{-1}$. This late-time signal can be interpreted as swept up nearby CSM of which a part interacts with a cloud of more distant CSM at 1150 days after peak brightness.

Strong late-time CSM interaction is very rare, and the only way to study them is by systematically searching for these events by monitoring a large group of known SNe using a survey such as ZTF. Discovering late-time interaction signals while they are still active is crucial for deep photometric and spectroscopic followup to constrain its properties. CSM trace the history of the progenitor system, and late-time interaction with distant CSM contains clues on the progenitor system as it was long before exploding. By measuring CSM properties such as its location, mass, geomtetry, and composition it is possible to learn about this history and put constraints on the type of progenitor system.

\end{abstracts}


\restoregeometry


