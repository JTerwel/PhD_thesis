\begin{acknowledgements}
%ZTF I & II & O4
Based on observations obtained with the Samuel Oschin Telescope 48-inch and the 60-inch Telescope at the Palomar Observatory as part of the Zwicky Transient Facility project. ZTF is supported by the National Science Foundation under Grants No. AST-1440341 and AST-2034437 and a collaboration including current partners Caltech, IPAC, the Oskar Klein Center at Stockholm University, the University of Maryland, University of California, Berkeley , the University of Wisconsin at Milwaukee, University of Warwick, Ruhr University, Cornell University, Northwestern University and Drexel University. Operations are conducted by COO, IPAC, and UW.

%GROWTH
This work was supported by the GROWTH project \citep{Kasliwal2019} funded by the National Science Foundation under Grant No 1545949.

%Fritz
The Gordon and Betty Moore Foundation, through both the Data-Driven Investigator Program and a dedicated grant, provided critical funding for SkyPortal.

%ZTFquery
The ztfquery code was funded by the European Research Council (ERC) under the European Union's Horizon 2020 research and innovation programme (grant agreement n°759194 - USNAC, PI: Rigault).

%DESI Legacy Imaging Surveys
The Legacy Surveys consist of three individual and complementary projects: the Dark Energy Camera Legacy Survey (DECaLS; Proposal ID \#2014B-0404; PIs: David Schlegel and Arjun Dey), the Beijing-Arizona Sky Survey (BASS; NOAO Prop. ID \#2015A-0801; PIs: Zhou Xu and Xiaohui Fan), and the Mayall z-band Legacy Survey (MzLS; Prop. ID \#2016A-0453; PI: Arjun Dey). DECaLS, BASS and MzLS together include data obtained, respectively, at the Blanco telescope, Cerro Tololo Inter-American Observatory, NSF’s NOIRLab; the Bok telescope, Steward Observatory, University of Arizona; and the Mayall telescope, Kitt Peak National Observatory, NOIRLab. Pipeline processing and analyses of the data were supported by NOIRLab and the Lawrence Berkeley National Laboratory (LBNL). The Legacy Surveys project is honored to be permitted to conduct astronomical research on Iolkam Du’ag (Kitt Peak), a mountain with particular significance to the Tohono O’odham Nation.

%SDSS
Funding for the SDSS and SDSS-II has been provided by the Alfred P. Sloan Foundation, the Participating Institutions, the National Science Foundation, the U.S. Department of Energy, the National Aeronautics and Space Administration, the Japanese Monbukagakusho, the Max Planck Society, and the Higher Education Funding Council for England. The SDSS Web Site is http://www.sdss.org/.

The SDSS is managed by the Astrophysical Research Consortium for the Participating Institutions. The Participating Institutions are the American Museum of Natural History, Astrophysical Institute Potsdam, University of Basel, University of Cambridge, Case Western Reserve University, University of Chicago, Drexel University, Fermilab, the Institute for Advanced Study, the Japan Participation Group, Johns Hopkins University, the Joint Institute for Nuclear Astrophysics, the Kavli Institute for Particle Astrophysics and Cosmology, the Korean Scientist Group, the Chinese Academy of Sciences (LAMOST), Los Alamos National Laboratory, the Max-Planck-Institute for Astronomy (MPIA), the Max-Planck-Institute for Astrophysics (MPA), New Mexico State University, Ohio State University, University of Pittsburgh, University of Portsmouth, Princeton University, the United States Naval Observatory, and the University of Washington.

%Pan-STARRS
The Pan-STARRS1 Surveys (PS1) and the PS1 public science archive have been made possible through contributions by the Institute for Astronomy, the University of Hawaii, the Pan-STARRS Project Office, the Max-Planck Society and its participating institutes, the Max Planck Institute for Astronomy, Heidelberg and the Max Planck Institute for Extraterrestrial Physics, Garching, The Johns Hopkins University, Durham University, the University of Edinburgh, the Queen's University Belfast, the Harvard-Smithsonian Center for Astrophysics, the Las Cumbres Observatory Global Telescope Network Incorporated, the National Central University of Taiwan, the Space Telescope Science Institute, the National Aeronautics and Space Administration under Grant No. NNX08AR22G issued through the Planetary Science Division of the NASA Science Mission Directorate, the National Science Foundation Grant No. AST-1238877, the University of Maryland, Eotvos Lorand University (ELTE), the Los Alamos National Laboratory, and the Gordon and Betty Moore Foundation.

%Gaia
This work has made use of data from the European Space Agency (ESA) mission
{\it Gaia} (\url{https://www.cosmos.esa.int/gaia}), processed by the {\it Gaia}
Data Processing and Analysis Consortium (DPAC,
\url{https://www.cosmos.esa.int/web/gaia/dpac/consortium}). Funding for the DPAC
has been provided by national institutions, in particular the institutions
participating in the {\it Gaia} Multilateral Agreement.

%iPTF
The intermediate Palomar Transient Factory project is a scientific collaboration among the California Institute of Technology, Los Alamos National Laboratory, the University of Wisconsin (Milwaukee), the Oskar Klein Centre, the Weizmann Institute of Science, the TANGO Program of the University System of Taiwan, and the Kavli Institute for the Physics and Mathematics of the Universe.

%ASASSN
ASAS-SN is funded by Gordon and Betty Moore Foundation grants
GBMF5490 and GBMF10501 and the Alfred P. Sloan Foundation
grant G-2021-14192.

%NOT \& ALFOSC
Based on observations made with the Nordic Optical Telescope, owned in collaboration by the University of Turku and Aarhus University, and operated jointly by Aarhus University, the University of Turku and the University of Oslo, representing Denmark, Finland and Norway, the University of Iceland and Stockholm University at the Observatorio del Roque de los Muchachos, La Palma, Spain, of the Instituto de Astrofisica de Canarias. The data presented here were obtained [in part] with ALFOSC, which is provided by the Instituto de Astrofisica de Andalucia (IAA) under a joint agreement with the University of Copenhagen and NOT.

%GTC \& OSIRIS
Based on observations made with the Gran Telescopio Canarias (GTC), installed at the Spanish Observatorio del Roque de los Muchachos of the Instituto de Astrofísica de Canarias, on the island of La Palma. This work is (partly) based on data obtained with the instrument OSIRIS, built by a Consortium led by the Instituto de Astrofísica de Canarias in collaboration with the Instituto de Astronomía of the Universidad Autónoma de México. OSIRIS was funded by GRANTECAN and the National Plan of Astronomy and Astrophysics of the Spanish Government.

%pypeit
This research made use of \textsc{PypeIt},\footnote{\url{https://pypeit.readthedocs.io/en/latest/}}
a Python package for semi-automated reduction of astronomical slit-based spectroscopy
\citep{pypeit:joss_pub, pypeit:zenodo}.

%Astropy
This work made use of Astropy:\footnote{http://www.astropy.org} a community-developed core Python package and an ecosystem of tools and resources for astronomy \citep{astropy:2013, astropy:2018, astropy:2022}.

%lmfit
Many people have contributed to lmfit. The attribution of credit in a project such as this is difficult to get perfect, and there are no doubt important contributions that are missing or under-represented here. Please consider this file as part of the code and documentation that may have bugs that need fixing.

Some of the largest and most important contributions (in approximate order of size of the contribution to the existing code) are from:
 Matthew Newville wrote the original version and maintains the project.
 
 Renee Otten wrote the brute force method, implemented the basin-hopping and AMPGO global solvers, implemented uncertainty calculations for scalar minimizers and has greatly improved the code, testing, and documentation and overall project.
 
 Till Stensitzki wrote the improved estimates of confidence intervals, and contributed many tests, bug fixes, and documentation.
 
 A. R. J. Nelson added differential\_evolution, emcee, and greatly improved the code, docstrings, and overall project.
 
 Antonino Ingargiola wrote much of the high level Model code and has provided many bug fixes and improvements.
 
 Daniel B. Allan wrote much of the original version of the high level Model code, and many improvements to the testing and documentation.
 
 Austen Fox fixed many of the built-in model functions and improved the testing and documentation of these.
 Michal Rawlik added plotting capabilities for Models.

 The method used for placing bounds on parameters was derived from the clear description in the MINUIT documentation, and adapted from J. J. Helmus's Python implementation in leastsqbounds.py.

 E. O. Le Bigot wrote the uncertainties package, a version of which was used by lmfit for many years, and is now an external dependency.

 The original AMPGO code came from Andrea Gavana and was adopted for lmfit.

 The propagation of parameter uncertainties to uncertainties in a Model was adapted from the excellent description at \url{https://www.astro.rug.nl/software/kapteyn/kmpfittutorial.html\#confidence-and-prediction-intervals}, which references the original work of: J. Wolberg, Data Analysis Using the Method of Least Squares, 2006, Springer.

Additional patches, bug fixes, and suggestions have come from Faustin Carter, Christoph Deil, Francois Boulogne, Thomas Caswell, Colin Brosseau, nmearl, Gustavo Pasquevich, Clemens Prescher, LiCode, Ben Gamari, Yoav Roam, Alexander Stark, Alexandre Beelen, Andrey Aristov, Nicholas Zobrist, Ethan Welty, Julius Zimmermann, Mark Dean, Arun Persaud, Ray Osborn, @lneuhaus, Marcel Stimberg, Yoshiera Huang, Leon Foks, Sebastian Weigand, Florian LB, Michael Hudson-Doyle, Ruben Verweij, @jedzill4, @spalato, Jens Hedegaard Nielsen, Martin Majli, Kristian Meyer, @azelcer, Ivan Usov, and many others.

The lmfit code obviously depends on, and owes a very large debt to the code in scipy.optimize. Several discussions on the SciPy-user and lmfit mailing lists have also led to improvements in this code.

Other software: numpy \citep{numpy}, pandas \citep{pandas_software, pandas_paper}, matplotlib \citep{matplotlib}, sncosmo \citep{sncosmo}, simsurvey \citep{simsurvey}

\end{acknowledgements}
