%!TEX root = ../main.tex
\documentclass[a4paper,oneside,12pt, class=Latex/Classes/PhDthesisPSnPDF, crop=false]{standalone}
\usepackage{setspace}
\begin{document}
\doublespacing
\chapter{Observing in the optical regime}
\label{chap:obs}

Astrophysicists face the unusual challenge of not being able to control their experiments. The universe is our laboratory but all we can do is observe the results while often not knowing the exact setup of the experiment. Models are made to explain and predict the behaviour of planets, stars, galaxies, etc. but ultimately observations are needed to compare against and test our models. My work relies heavily on observational data, and in this chapter I will introduce the telescopes and instruments that are at the basis of this thesis. I will also give a general overview of what to consider when planning observations and different types of observations that can be done in the optical regime. \color{red}add refs to sections, reduction and analysis in a separate chapter probably \color{black}


\section{Telescopes}
Most of the data used in this thesis comes from the Zwicky Transient Facility (ZTF), and follow-up observations have been made using the Nordic Optical Telescope (NOT), and the Gran Telescopio Canarias (GTC), which will be introduced below. Some additional data comes from other sources, which we list for completeness. \color{red}ow subsection or list here \color{black}

\subsection{Zwicky Transient Facility}
The Zwicky Transient Facility (ZTF) is an optical large-sky survey observing the entire northern night sky above Dec $\sim-30$\degree every 2-3 nights in three broadband optical filters \ztfg\ztfr\ztfi, which are very similar to the well-known SDSS \ztfg\ztfr\ztfi\ filters. The efficiency of these filters is plotted as a function of wavelength in Fig. \ref{Optical_elements_plot}. The survey saw first light in October 2017 and the survey formally began scientific operation in March 2018 and has been running continueously until the time of writing this document.

The observations are made using the $48\arcsec$ aperture Schmidt-type design Samuel Oschin Telescope, which is based at the Palomar Observatory in Southern California. Each exposure, lasting 30 s, can go a limiting magnitude of $\sim20.5$ mag and covers an area of $\sim47$ deg$^2$ at a resolution of of $1.01\arcsec$ per pixel. The camera is divided in a $4\times4$ grid of CCDs, each of which have 4 readout channels called quadrants. This result in each observation producing 64 separate images, each with their own quadrant identifier (qid). Similarly, the observed region of the sky is divided into different telescope pointings called fields (identified using fid) to ensure that the same region of the sky is observed in the same way each time, aiding with the reduction of the data. This results in each combination of filter, fid, and qid being a set of observations of a particular part of the sky using speciic setup. \color{red} Add all the usual ZTF references \color{black}

\begin{figure}
    \centering
    \includegraphics[width=\textwidth]{../Images/chapter_2/transmissions.png}
    \caption{Throughput as a function of wavelength of the different filters used to gather the bulk of the data in this thesis \ztfg filters are shown in green, \ztfr in orange, \ztfi in red, and the different telescopes are shown with different line styles (Continuous for ZTF, dashed for NOT, dot-dashed for GTC). The SDSS filters (dotted lines) are shown for comparison. For the grisms the wavelength ranges are shown as only, not their efficiency as function of wavelength. \color{red}Plot needs to be finished, see to-do list in the notebook \color{black}}
    \label{Optical_elements_plot}
\end{figure}


\subsection{Nordic Optical Telescope}
The Nordic Optical Telescope (NOT) is a 2.56 m telescope located at Roque de Los Muchacos Observatory in La Palma, Spain. \color{red} Add elevation and coords? \color{black} It hosts several instruments for observing in the optical and near infrared, both for imaging and spectroscopy. The main instrument is the Alhambra Faint Object Spectrograph and Camera (ALFOSC), which was used to obtain the data used in this thesis. I will only discuss the parts relevant to this thesis, further details on this instrument and  details on the other instruments can be found at \footnote{\url{https://not.iac.es}}.

ALFOSC is a versatile instrument mounted in cassegrain that can be used for imaging, spectroscopy, and (spectro)polarimtery. As there are several wheels equipped to hold a variety of optical elements, the instruments can switch quickly between different setups between observations. The images can cover up to $6.4\arcmin\times6.4\arcmin$ per exposure at a resolution of $0.2138\arcsec$ per pixel. In this thesis filters 120 (\ztfg$\arcmin$), 110 (\ztfr$\arcmin$), and 111 (\ztfi$\arcmin$) are used for spectroscopy. For spectroscopy grism 4 is used together with a $1.0\arcsec$ slit if the seeing was $\leq1.3\arcsec$ or a  $1.3\arcsec$ slit if the seeing was $\geq1.3\arcsec$. For some spectra an order-blocking filter (WG345) is used as well to avoid second order diffracted blue light to overlap with first order diffracted red light on the detector. Details on these optical elements are given in Table \ref{NOT_optic_elems}, and they are shown in Fig. \ref{Optical_elements_plot}.

\begin{table}
    \centering
    \caption{Optical elements used for observations taken with NOT/ALFOSC and GTC/OSIRIS+. \color{red}Finish it, check if R1000R also has an effective range (think I used it all though)\color{black}}
    %\resizebox{\textwidth}{!}{%Scale table to page width
    	\begin{tabular}{ccccc}
    		\hline
    		\hline
    		Filter & $\lambda_\text{center}$ (\AA) & FWHM (\AA) & T$_\text{max}$\\
    		\hline
    		\ztfg$\arcmin$ NOT & 4800 & 1450 & 0.92\\
    		\ztfr$\arcmin$ NOT & 6180 & 1480 & 0.90\\
    		\ztfr$\arcmin$ GTC & 6410 & 1760 & 0.94\\
    		\ztfi$\arcmin$ NOT & 7710 & 1710 & 0.91\\
    		WG345 & 3560 & - & 0.88\\
    		\\
    		\hline
    		\hline
    		Grism & $\lambda$ range (\AA) & resolution (\AA / pixel) & Orientation\\
    		\hline
    		\#4 & 3200$^*$ - 9600 & 3.3 & vertical\\
    		R1000R & 5100 - 10000 & 2.62 & horizontal\\
    		\\
    		\hline
    		\hline
    		slit & Telescope & Orientation\\
    		\hline
    		1.0$\arcsec$ & NOT & horizontal & \\
    		1.0$\arcsec$ & GTC & vertical & \\
    		1.3$\arcsec$ & NOT & horizontal & \\
    		\hline
    	\end{tabular}
    %}
    \begin{flushleft}
    	$^*$ The detector response is limited at low wavelengths, so in practice a lower limit of 4000 \AA\ is used for faint targets.
    \end{flushleft}
    \label{NOT_optic_elems}
\end{table}


\subsection{Gran Telescopio CANARIAS}
The Gran Telescopio CANARIAS (GTC) is a 10.4 m telescope at the Roque de los Muchachos Observatory in La Palma, Spain, and is the largest optical / near infrared telescope on the island. Its primary mirror is made up from 36 hexagonal pieces creating an effective collection area of 73 m$^2$, ideal for observing very faint targets. The GTC can host up to six instruments at a time in various focal positions, allowing for a large variety of observations to be made. One of the most commonly used instruments is OSIRIS+, the upgraded version of OSIRIS, Optical System for Imaging and low-Intermediate-Resolution Integrated Spectroscopy.

OSIRIS+ has an unvignetted field-of-view of $7.8\arcmin\times7.8\arcmin$ at a resolution of $0.254\arcsec$ per pixel. Since the standard readout has $2\times2$ binning, the resolution can be increased to $0.127\arcsec$ if so desired. Like ALFOSC, this isntrument is also built to easily switch between different setups between observations. For photometry the $r\arcsec$ filter is used in this thesis, and for spectroscopy the R1000R grism with a $1.0\arcsec$ slit is used. Details on these optical elements are given in Table \ref{NOT_optic_elems}, and they are shown in Fig. \ref{Optical_elements_plot}. \color{red} Add refs \color{black}


\subsection{Other observations}
Small amounts of data coming from other telescopes and surveys are presented in this thesis as well. This includes a follow-up observation of SN 2019ldf in \ztfg and \ztfr using the ESO Faint Object Spectrograph and Camera version 2 (EFOSC2) imaging spectrograph on the ESO New Technology Telescope (NTT) in La Silla, Chile as part of the extended Public ESO Spectroscopic Survey of Transient Objects+ (EPESSTO+).

To complement ZTF data of several SNe observations from the Panoramic Survey Telescope and Rapid Response System (Pan-STARRS, (intermediate) Palomar sky Survey (PTF, / iPTF), All Sky Automated Survey for SuperNovae (ASASSN), Asteroid Terrestrial-impact Last Alert System (ATLAS), Global Astrometric Interferometer for Astrophysics (Gaia), and Wide-Field Infrared Survey Explorer (WISE) are used. \color{red} add refs \color{black}



\section{General considerations for observing}
To observe astronomical objects one has to consider several things. Assuming that the location or region on the sky that we are interested in to observe is already known, as well as the desired type of observation, an observing plan can be made. A well constructed observing plan should give the best quality data possible while making efficient use of the resources available.


\subsection{Location}
Although this is normally already done before constructing a telescope, the first thing to consider is the location from where to observe. When purely aiming for the best observations possible, there are three main things to consider when chosing a location:
\begin{itemize}
	\item {Weather: Clear skies for most nights throughout the year, stable conditions, and low atmospheric distortion (seeing) are vital to ensure good quality data on a regular basis. By going higher above sea level, lower hanging clouds can be avoided while simultaneously decreasing the amount of air starlight has to travel through to reach the detector.}
	\item {Light pollution: The darker the sky, the fainter objects can be observed. Artificial light sources from humans greatly hinder the observation of faint objects by outshining them many times over. Even the the prescence of a (partially) illuminated moon greatly changes the depth that can be reached. For this reason many observatories have (inter)national laws to control the ligth pollution and ensure good quality data can be obtained.}
	\item {Target observability: The target needs to be high enough above the sky for long enough during the night for observations to be made. Additionally, the closer to zenith an observation is made, the better quality data as it decreases the amount of atmosphere between the target and telescope. The atmosphere reduces the data quality seeing, broadband absorbtion (clouds, dust), narrowband interference (tellurics, skylines), and achromatic diffraction (different colours diffracting differently when entering the atmosphere at an angle, \color{red} Check name and cite that specific paper \color{black}) among others.}
\end{itemize}

Combined, this means that observatories should be located on top of high mountains that are in areas with stable and clear weather, with as small a population nearby as possible while still being accessible enough for transporting materials and observing staff. One of the best locations in the world that meets these requirements is Roque de los Muchachos on La Palma, a small Spanish island in the Atlantic ocean off the coast of Morocco. At around 2300 m above sealevel, the telescopes are built on the highest peak of the mountain, far from most communities on the island which are much closer to sea level, and the temperate climate ensures good sky conditions for most nights around the year. Additionally, the government has put laws in place to minimize light pollution, e.g. by limiting the use of street lights and restricting flight patterns over the observatory.


\subsection{Telescope, instrument, mode, and setup}
Depending on the type of observations and the brightness of the target there is a choice of telescopes to be used. Telescope, intsrument, observing mode, and desired setup(s) have to be considered together, as some choices affect other ones.

Bigger telescopes can observe fainter targets, but it is also more difficult to obtain observing time. On the other hand, smaller telescopes are less oversubscribed (a measure of requested versus available observing time), but are more limited in observation depth even with longer exposure times. Smaller telescopes are however more ideal for brighter targets that will instantly saturate the detector of a larger telescope.

Secondly, different instruments, which are often telescope specific, have different observing capabilities. Even though ALFOSC and OSIRIS+ can both do photometry and spectroscopy, there are still differences in data quality and resolution even if the same object is observed at the same time. Spectroscopy and photometry are very standard observing modes, and most telescopes have an instrument can offer this. However, for polarimetric observations OSIRIS+ cannot be used but ALFOSC can, limiting the options for this mode of observation.

Lastly, the specific setup has to be considered as well. For photometry, which filters are desired? If a very specific or rare filter is required this may again limit the options of telescopes and instruments. For spectrocopy there are other choices, such as fiber or slit spectroscopy, different gratings or grisms depending on the desired resolution and wavelength range, neutral density filters to observe targets that are otherwise too bright for the instrument, and order-blocking filters to remove blue light from red parts of the spectrum.


\subsection{Night plan}
\color{red} Add a NOT night plan as a plot? Could be a good visualization and I make them anyway so its easy to screenshot \color{black}

Lastly, it is recommended to have planned what to observe when to avoid losing observing time during the night. While most proposals already have a list of targets and standard stars to observe and exposure times when they are submitted, the detailed plan is usually made mere hours before the night starts as it depends on e.g. the current weather conditions and stability, which targets have already been observed a previous night, specific time constraints (e.g. for transits), and target priority. Calibration images need to be taken as well, and although some can be taken during the day others can only be taken during a short window in twilight, or have to be taken directly before or after the target. All of these things need to be taken into concideration when trying to maximize the time used to expose and observe targets, and minimize the overheads from e.g. positioning, target acquisition, and readout.

Time spent repositioning the telescope can be reduced by trying to find the path between targets that minimizes telescope and dome movement throughout the night. The time spent acquiring the target depends on the observing mode but also on the experience and tiredness of the observer. Photometry observes a field, so usually a small offset is not disastrous for the science. Spectroscopy takes longer as the correct target needs to be identified and placed in the slit or fiber before the exposure can start. Readout times are specific to each detector, though if only a part of the CCD is needed windowing and binning during readout can shorten this significantly. This can be especially important in cases where multiple shorter exposures are taken instead of a single long one. This can be done to e.g. reduce cosmic ray interference, avoid overexposure of a bright source close to a fainter target, or for constructing time series.

\begin{figure}
    \centering
    \includegraphics[width=\textwidth]{../Images/chapter_2/visplot.png}
    \caption{Night plan for the NOT on the night of XXX. Targets are plotted with their altidute as a function of universal standard time and local stellar time on top. The target priority has been colour coded, with the coloured bars showing the amount of time each observation is expected to take. Green targets have already been completed, and the red vertical line shows the current time. Not all targets fit into the schedule but are still shown in case the plan has to be ammended during the night.\color{red} Temp version, find and make a proper one \color{black}}
    \label{visplot}
\end{figure}

Nothing is certain during the night. Weather conditions can change or not meet the conditions required for some observations, technical problems can occur, or observations might go so smoothly that they are completed faster than expected. A flexible schedule with a priority list and backup targets helps to adapt to these situations as fast as possible. After all, an idling telescope in (half-)decent observing conditions is a waste of resources. Fig.~\ref{visplot} shows an example night plan for the NOT.



\section{Types of obeservations}
All optical observations are, in esssence, images taken by a camera. Light falls onto a pixel on the CCD, and frees some amount of electrons. The more light that hits the pixel, the more electrons are freed. At readout these electrons are counted per pixel, or group of pixels if binning is applied, and turned into a digital number called a count. During this process there are contributions from different noise sources, but as long as the total count rate is in the linear regime of the CCD, i.e. there is a linear relation between the received flux and final count, it is possible to calculate the flux by using calibration images. The different types of calibration images are described below, but their usage is explained in section \ref{reduction} when discussing image reduction.

\color{red} reference some books or so about observing techniques or something \color{black}


\subsection{Photometry}
\begin{figure}
    \centering
    \includegraphics[height=0.8\textheight]{../Images/chapter_2/phot_and_spec_example.png}
    \caption{Image and partial spectrum of SN 2024nqr (left) and SN 2024pgd (right), two SNe Ia active simultaniously in the same galaxy. The image was taken without a filter and used to align the 1.0$\arcsec$ slit (horizontal dashed lines) over both SNe. The resulting spectrum, taken with grism \#4, shows three traces as white vertical stripes. The outer two line up with the two SNe, while the middle trace is from the host galaxy edge in the slit (vertical dotted lines for guidance). The horizontal lines in the spectrum are sky lines coming from atmospheric emission. This data was taken with NOT/ALFOSC on the night of 28 July 2024 while testing an experimental rapid response mode (RRM, credit: Samuel Grund S\o rensen). \color{red}check AT/SN status weirdness \color{black}}
    \label{phot_spec_example}
\end{figure}


Photometry is one of the simplest observing modes as it is just taking a photo of a part of the sky. The difference with the camera in a phone is that the telescope instrument is much more sensitive. The top of Fig.~\ref{phot_spec_example} shows a raw photometric image, taken with NOT/ALFOSC without the use of a filter. The images are monochromatic, i.e. they only have a value for the intensity. For colourful images multiple observations have to be made in different filters and combined to represent different colours. Faint objects can be observed by increasing the exposure time in a single image, or stacking multiple images together to increase the effective exposure time. When stacking images it is common practice to dither the telescope: applying a small offset between exposures to ensure that the target hits a different part of the CCD, avoiding issues with bad pixels ruining otherwise good observations. While this decreases the effective size of the fully stacked image, as long as the edges are not needed there is no issue.


\subsection{Spectroscopy}
Spectroscopy goes one step beyond just taking a photo. Assuming that this is slit spectroscopy, instead of a filter to select a wavelength range to observe now a slit restricts the observable region of the sky to a narrow band along one axis of the detector (e.g. horizontal). After the slit the light hits a grating or grism (a grating and prism combined) which diffracts the light based on wavelength across the second axis of the detector (vertical). The rule density on the grating / grism dictates the wavelength spread of the light: the more rules per unit distance, the bigger the diffraction, and the higher the spectral resolution of the resulting image. The tradeoff is that a smaller part of the spectrum can be observed at a time, and there is less light being received per pixel which reduces the SNR unless the exposure time is increased to account for this. Any point-like source that is observed becomes a line in the spectral direction, called a trace.

There is some freedom in the orientation of the slit. This is called the position angle of the slit. If there are multiple targets near each other, and they can be in the slit at the same time, the required position angle can be calculated from the two target positions. If there is a single target to be observed the position angle can be anything, but usually the parallactic angle is chosen. In this orientation the slit is perpendicular to the horizon, as this minimizes losses from the achromatic diffraction due to the atmosphere at different wavelengths.

The bottom panel of Fig.~\ref{phot_spec_example} shows a section of the spectrum taken of the image in the top panel. The two SNe are drawn out into vertical traces and a third trace belonging to the edge of the host galaxy can be seen in the middle. The horizontal lines are sky emission lines, and while these can technically be used to estimate the conversion from pixel position to wavelength, standardized arc frames will result in a much better wavelength calibration (see below).



\subsection{Calibration: Bias}
The first calibration image is the bias, which is made by reading out the CCD without exposing. The resulting image contains the amount of counts that will be in every exposure regardless of what has been observed or with what exposure time. In other words, measuring the bias can be tought of as measuring the offset to correct for in every other image.

%$B_{ij}$ is the so-called bias, the measured pixel response that is in every observation regardless of the exposure time and amount of light hitting the pixel. The easiest way to measure this value is to take bias frames: observe with an exposure time of 0 with no light hitting the CCD. $F$ and $t$ are 0 which reduces Eq.~\ref{CCD_response} to $R_{ij}(0, 0) = B_{ij}$

\subsection{Calibration: Dark}
Any detector that is not at a temperature of 0 K will have some amount of noise due to thermal effects. This can free electrons in pixels over time, creating a dark current and increasing the noise over time. The effect can be measured by exposing for the same amount of time as the science images taken, but without letting any light hit the CCD. This is called a dark frame.

As this is a thermal effect, it can be reduced to negligible amounts by cooling the instrument. This saves precious observing time, as otherwise dark frames would ideally have to be taken at the same temperature as the target was observed, which is easiest to do directly after the science exposure. By cooling the detector with e.g. liquid nitrogen this noise source can be avoided instead of having to correct for, saving time and the amount of images that need to be taken in the process.

%Next is $D_{ij}$, the noise that increases with longer exposure times. This is thermal noise, and as its name suggests it is more impactful the higher the temperature $T$ is in the CCD: $D_{ij} = D_{ij}(T)$. Due to this there are actually two ways to remove this source of noise. One can take Dark frames by exposing for some amount of time but making sure no light hits the detector, $R_{ij}(0, t) = B_{ij} + D_{ij} \times t$, and extract $D_{ij}$. This takes however a lot of time, and it is often more efficient to cool the CCD down to very low temperatures (e.g. with liquid nitrogen) to ensure $D_{ij}$ can be neglected.

\subsection{Calibration: Flats}
The amount of light that the CCD receives is converted into a digital number, but there is no guarantee that this conversion rate is the same for each pixel. This can be due to intrinsic differences between the pixels, or outside effects such as dust reducing the amount of light recieved on a part of the detector. To correct for this an evenly illuminated field has to be observed called flats or flatfields. By ensuring that each pixel receives the same amount of light, the different counts will reflect the varying responses per pixel.

While any evenly illuminated object can be used for this, such as the the inside of the telescope dome to create dome flats. A more perfect evenly lit source however is the sky, and using this sky flats can be taken. While it is usually too bright during the day and the CCD will saturate even with the narrowest filter and shortest exposure time, there is a window during twilight where the sky is darker but not dark enough to observe stars yet. As a general rule, narrowband filters need a brighter sky and in the evening need to be done before the broadband filters. After that, assuming similar efficiencies between filters, blue filters need brighter skies than red filters, forcing a specific order in which the sky flats need to be taken during the short window where this is possible. Of course if flats are taken in the morning the order has to be reversed.

%Lastly, to measure $A_{ij}$ flatfield images are needed. During a science exposure the amount of light hitting a pixel is dependent of the brightness of the source at that location, $F = F_{ij}$. But as not every pixel may have the same response the measured values need to be normalized before different pixels can be compared. This is done by observing an evenly lit background where $F_{ij}$ is the same for every pixel. This can be done by observing the inside of the dome (dome flats), although the twilight sky (sky flats) are usually preferred as they are more evenly lit across the CCD.

\subsection{Calibration: Arc}
In spectroscopy, one of the CCD axes is spectral with red light at one end and blue light at the other end of the detector. To know where on the detector each wavelength falls, so-called arc frames are needed. These are taken by observing a lamp filled with a known set of elements (e.g. He, Ne, or TH and Ar). The wavelengths of the emission lines are known very precisely, and by matching these with the observed lines in the arc image a pixel-to-wavelength conversion can be found, called the wavelength solution.

%For spectroscopy another type of calibration is needed: Wavelength calibration. This is done with arc frames: observing the known emission lines of several elements using lamps in the instrument (e.g. He, Ne, ThAr lamps) using the same setup as the science observation. The resulting pattern of emission lines is known and can be used to map the pixel position in the spectral direction to a wavelength, called the wavelength solution.


\end{document}